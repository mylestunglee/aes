Scheduling arises in many decision processes and has a wide range of practical applications, such as in healthcare. Scheduling problems are modelled using mathematics, where optimisers find solutions to large scheduling problems quickly. Problems and solutions are often complex with inaccessible formulations, resulting in users interpreting solvers and solutions as black boxes. This hinders the ability to understand why a schedule is reasonable, that is critical for decision-making.
\linespace
We build an argumentation-supported tool, namely, \emph{\toolname} that explains any makespan schedule with clarity. We will explore theoretical and practical considerations of applying argumentation to makespan scheduling.