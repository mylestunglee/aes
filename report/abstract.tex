Scheduling arises in many decision processes and has a wide range of practical applications, such as in healthcare. Scheduling problems are mathematically modelled, where optimisers find solutions to large scheduling problems quickly. Problems and solutions are often complex with non-accessible formulations, resulting in users interpreting solvers and solutions as black-boxes. Users require a means to understand why a schedule is reasonable, which is hindered by black-box interpretations.
\linespace
We build an argumentation-supported tool, namely, \emph{\toolname} that explains any makespan schedule easily with clarity. We will explore practical considerations of applied argumentation to makespan scheduling and theoretical generalisations of argumentation, as in interval scheduling.