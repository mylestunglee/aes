Scheduling arises in countless decision processes and has a wide range of practical applications, such as in healthcare. Scheduling problems are modelled using mathematics, where optimisers can find solutions to large scheduling problems quickly. Such optimisers are often complex with non-accessible formulations of scheduling, resulting in users interpreting solvers and solutions as black-boxes. Users require a means to understand why a schedule is reasonable, which is hindered by black-box interpretations.
\linespace
We present a tool, \emph{\toolname} that explains any makeshift schedule easily with clarity. We will explore practical considerations of applied argumentation and extensions to makeshift scheduling, highlighted by the tool.